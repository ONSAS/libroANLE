% ------------------------------------
% paquetes de formato

% paquetes basicos
\usepackage[utf8]{inputenc}
\usepackage[spanish]{babel}

%\usepackage{epsfig}
\usepackage{graphicx}
\usepackage{epstopdf}

\usepackage{amssymb,amsmath,amsfonts,amsthm}%,stmaryrd,latexsym,bbold,psfrag,fontenc,dsfont}
\usepackage{bm}
\usepackage{float}
\usepackage{booktabs}
\usepackage{lscape}
\usepackage{wrapfig}

% color
\usepackage[usenames]{color}
\color{black} % el paquete color define el color por defecto Black cmyk 0 0 0 1 que no es totalmente negro
\usepackage[skins]{tcolorbox}

\usepackage{subfig}

\usepackage{hyphenat}
%\usepackage[hidelinks,bookmarks=false]{hyperref}
\usepackage[plainpages=false]{hyperref}

% diagramacion de hoja
\usepackage[a5paper,top=16.5mm,bottom=12.5mm,left=16.5mm,right=12.5mm,portrait]{geometry}

%\usepackage[cam,a4,center]{crop}
\usepackage[frame,a4,center]{crop}

% fuente
\usepackage{libertine}
\usepackage{libertinust1math}
\usepackage[T1]{fontenc}
% ------------------------------------


% -----------------------------
% estilo de bibliografia
\usepackage{natbib}
\bibliographystyle{../bib/apalike-es}
%\bibliographystyle{apalike}
% -----------------------------

% ----------------------------------------------------
% ----- encabezados ------------
\usepackage{fancyhdr,fancybox}

\pagestyle{fancy}

% with this we ensure that the chapter and section headings are in lowercase.
\renewcommand{\chaptermark}[1]{\markboth{\chaptername\ \thechapter.\hspace{1ex}#1}{}}
\renewcommand{\sectionmark}[1]{\markright{Sección \thesection.\hspace{1ex}#1}}
\fancyhf{} % delete current header and footer
\fancyhead[LE,RO]{\small\sffamily\bfseries\thepage}
\fancyhead[LO]{\small\sffamily\bfseries\nouppercase{\rightmark}}
\fancyhead[RE]{\small\sffamily\bfseries\nouppercase{\leftmark}}
\renewcommand{\headrulewidth}{0.6pt}
\renewcommand{\footrulewidth}{0pt}
\addtolength{\headheight}{3.6pt} % space for the rule
\fancypagestyle{plain}{%
	\fancyhead{} % get rid of headers on plain pages
	\renewcommand{\headrulewidth}{0pt} % and the line
}
% ----------------------------------------------------


\usepackage[normalem]{ulem}
\usepackage[labelfont={small,sf,bf}]{caption}
\usepackage{enumitem}

% para setear interlineado
\usepackage{setspace}

% --------------------------------
% ----- algoritmos y codigos -----

% algoritmos-pseudocodigos
\usepackage{algorithmic}
\usepackage{algorithm}
%\usepackage{algcompatible} %\usepackage{algpseudocode}
\makeatletter
\renewcommand*{\ALG@name}{Algoritmo}
\makeatother
% ----------------------

% marca de agua: EN REVISION
\usepackage[printwatermark]{xwatermark}
\newwatermark[pages=1-100,color=red!10,angle=45,scale=2.0,xpos=0,ypos=0]{En revisión.}


% ----------------------
% codigos embebidos
\usepackage[numbered,framed]{matlab-prettifier}
%
\let\ph\mlplaceholder % shorter macro
\lstMakeShortInline"

\lstset{
	style              = Matlab-editor,
	basicstyle         = \footnotesize\mlttfamily,
	language           = Octave,
	escapechar         = ",
	mlshowsectionrules = true,
}
\renewcommand\lstlistingname{Código}
% ----------------------



% ----------------------------------------------
% seteos varios
% ----------------------------------------------

\usepackage{bigfoot} % to allow verbatim in footnote

\usepackage{titlesec}
\usepackage{titletoc}

%\titleformat{\chapter}[display]{\Huge\sffamily\bfseries}{\chaptertitlename~\thechapter}{2ex}{}
\titleformat{\section}   {\large\sffamily\bfseries}{\thetitle.}{1ex}{}
\titleformat{\subsection}{\small\sffamily\bfseries}{\thetitle.}{1ex}{}
\titleformat{\subsubsection}{\small\sffamily\bfseries}{}{1ex}{}

\titlecontents{chapter}[1.3em]
{\vspace{1.1em}\sffamily\bfseries}
{\contentslabel{1.3em}}
{\hspace*{-1.3em}}
{\titlerule[1pt]\contentspage}
\titlecontents{section}[3.5em]
{\small\sffamily} % note that 3.5 = 1.3 + 2.2
{\contentslabel{2.2em}}
{\hspace*{-2.2em}}
{\titlerule*[1pc]{.}\contentspage}
\titlecontents{subsection}[6.4em]
{\small\sffamily} % note that 6.4 = 3.5 + 2.9
{\contentslabel{2.9em}}
{\hspace*{-2.9em}}
{\titlerule*[1pc]{.}\contentspage}

% --------------------------------------------------------
% Cuadros en ejemplos y ejercicios
% -------------------------------------------------------
\usepackage{framed}

\renewcommand{\FrameCommand}{\shadowbox}

\definecolor{shadecolor}{rgb}{0.8,0.8,0.8}

\setlength{\fboxsep}{0.5em}
%\setlength{\FrameSep}{0.0mm}
%\setlength{\FrameRule}{1pt}
% --------------------------------------------------------

%
\usepackage{pdfpages}



% ----------   paquetes para diagrama de flujo apendice ------------
\usepackage{pgf,tikz}
\usetikzlibrary{arrows,backgrounds,shapes.geometric,calc}
%\usetikzlibrary{arrows,snakes,backgrounds,shapes.geometric,calc}
\listfiles
% -------------------------------------


\usepackage{relsize}

\usetikzlibrary{arrows.meta}
\tikzset{%
	>={Latex[width=2mm,length=2mm]},
	% Specifications for style of nodes:
	base/.style = {rectangle, rounded corners, draw=black,
		minimum width=4cm, minimum height=1cm,
		text centered, font=\sffamily},
	activityStarts/.style = {base, fill=blue!30},
	startstop/.style = {base, fill=red!30},
	activityRuns/.style = {base, fill=green!30},
	process/.style = {base, minimum width=2.5cm, fill=orange!15,
		font=\ttfamily},
}


% esto resuelve problema tikz con babel
\usepackage{etoolbox}
\AtBeginEnvironment{tikzpicture}{\shorthandoff{>}\shorthandoff{<}}{}{}



%
% --------------------------------------------------------
% para figuras con tikz e inkscape

\newcommand*\circled[1]{\tikz[baseline=(char.base)]{
		\node[shape=circle,draw,inner sep=2pt] (char) {#1};}}

% Comandos figuras inkscape
%\newcommand{\executeiffilenewer}[3]{%
%\ifnum\pdfstrcmp{\pdffilemoddate{#1}}%
%{\pdffilemoddate{#2}}>0%
%{\immediate\write18{#3}}\fi%
%}
%\newcommand{\includesvg}[1]{%
%\executeiffilenewer{#1.svg}{#1.pdf}%
%{inkscape -z -D --file=#1.svg %
%--export-pdf=#1.pdf --export-latex}%
%\input{#1.eps_tex}%
%}

% agrega ruta a figuras
\graphicspath{{../fig/}}
% -------------------------------------------
%
%
\newtheoremstyle{miestilo}% <name>
{0pt}% <Space above>
{0pt}% <Space below>
{\normalfont}% <Body font>
{0pt}% <Indent amount>
{\small\sffamily\bfseries}% <Theorem head font>
{\quad}% <Punctuation after theorem head>
{0pt}% <Space after theorem head>
{}% <Theorem head spec (can be left empty, meaning `normal')>

\addto{\captionsspanish}{\renewcommand*{\tablename}{Tabla}}

\theoremstyle{miestilo}
\newtheorem{SHteorema}{Teorema}[chapter]
\newtheorem{SHcorolario}[SHteorema]{Corolario}
\newtheorem{SHhipotesis}[SHteorema]{Hipótesis}
\newtheorem{SHlema}[SHteorema]{Lema}
\newtheorem{SHnotacion}[SHteorema]{Notación}
\newtheorem{SHdefinicion}[SHteorema]{Definición}
\newtheorem{SHejemplo}[SHteorema]{Ejemplo}
\newtheorem{SHejercicio}[SHteorema]{Ejercicio}
\newtheorem{SHobservacion}[SHteorema]{Observación}
\newtheorem{axioma}{Axioma}
%
\newenvironment{teorema}[1][]{\begin{SHteorema}[#1]\begin{framed}}{\end{framed}\end{SHteorema}}
\newenvironment{corolario}[1][]{\begin{SHcorolario}[#1]\begin{framed}}{\end{framed}\end{SHcorolario}}
\newenvironment{hipotesis}[1][]{\begin{SHhipotesis}[#1]\begin{framed}}{\end{framed}\end{SHhipotesis}}
\newenvironment{lema}[1][]{\begin{SHlema}[#1]\begin{framed}}{\end{framed}\end{SHlema}}
\newenvironment{definicion}[1][]{\begin{SHdefinicion}[#1]\begin{framed}}{\end{framed}\end{SHdefinicion}}
\newenvironment{notacion}[1][]{\begin{SHnotacion}[#1]\begin{framed}}{\end{framed}\end{SHnotacion}}
\newenvironment{ejemplo}[1][]{\begin{SHejemplo}[#1]\begin{framed}}{\end{framed}\end{SHejemplo}}
\newenvironment{ejercicio}[1][]{\begin{SHejercicio}[#1]\begin{framed}}{\end{framed}\end{SHejercicio}}
\newenvironment{observacion}[1][]{\begin{SHobservacion}[#1]\begin{framed}}{\end{framed}\end{SHobservacion}}

\newcommand{\subtitulo}[1]{~\\\noindent{\small\sffamily\bfseries #1}\vspace{2mm}}

\usepackage{lineno}

%\definecolor{miblue}{rgb}{0,0.1,0.5}%
\definecolor{miblue}{rgb}{0,0.1,0.38}%
%
\definecolor{colorbordecajas}{rgb}{0,0.1,0.4}%
\definecolor{colorfondocajas}{rgb}{.86,.91,.996}%
%
%
% -----------  cajas ----------------------------
\newcommand{\cajaactividad}[1]{
\begin{center}
\begin{tcolorbox}[enhanced,width=0.9\textwidth,drop fuzzy shadow southwest,colframe=colorbordecajas,colback=colorfondocajas,title=\textbf{Actividad}]
#1
\end{tcolorbox}
\end{center}
}

\definecolor{colorbordecajasconcepto}{rgb}{0.1,0.33,0.01}%
%\definecolor{colorfondocajasconcepto}{rgb}{.73,.95,.65}%
\definecolor{colorfondocajasconcepto}{rgb}{.9,.99,.8}%

\newcommand{\cajaconcepto}[2]{
	\begin{center}
		\begin{tcolorbox}[enhanced,width=0.9\textwidth,drop fuzzy shadow southwest,colframe=colorbordecajasconcepto,colback=colorfondocajasconcepto,title=\textbf{#1}]
			#2
		\end{tcolorbox}
	\end{center}
}
% ---------------------------------------------------------
%
%
\titleformat{\chapter}
{\normalfont\Large\color{miblue}\bfseries}{\color{miblue}\chaptername\ \thechapter:}{1em}{}[{\color{miblue}\titlerule[0.8pt]}]


%\usepackage{forest}

%\definecolor{foldercolor}{RGB}{124,166,198}
%
%\tikzset{pics/folder/.style={code={%
%			\node[inner sep=0pt, minimum size=#1](-foldericon){};
%			\node[folder style, inner sep=0pt, minimum width=0.3*#1, minimum height=0.6*#1, above right, xshift=0.05*#1] at (-foldericon.west){};
%			\node[folder style, inner sep=0pt, minimum size=#1] at (-foldericon.center){};}},
%	pics/folder/.default={20pt},folder style/.style={draw=foldercolor!80!black,top color=foldercolor!40,bottom color=foldercolor}
%}
%
%\forestset{is file/.style={edge path'/.expanded={%
%			([xshift=\forestregister{folder indent}]!u.parent anchor) |- (.child anchor)},
%		inner sep=1pt}, this folder size/.style={edge path'/.expanded={%
%			([xshift=\forestregister{folder indent}]!u.parent anchor) |- (.child anchor) pic[solid]{folder=#1}}, inner ysep=0.6*#1},
%	folder tree indent/.style={before computing xy={l=#1}},
%	folder icons/.style={folder, this folder size=#1, folder tree indent=3*#1},
%	folder icons/.default={12pt},
%}



% archivo de definiciones
% ---- Some commands definitions file for latex -------

% --- nombres propios ---
\def\Tikhonov{Tikhonov} 
\def\Young{Young}
\def\Poisson{Poisson}
\def\Cauchy{Cauchy}

\def\bbA{{\mathbb A}} 
\def\bbB{{\mathbb B}} 
\def\bbC{{\mathbb C}} 
\def\bbE{{\mathbb E}} 
\def\bbF{{\mathbb F}} 
\def\bbI{{\mathbb I}} 
\def\bbL{{\mathbb L}} 
\def\bbN{{\mathbb N}} 
\def\bbP{{\mathbb P}} 
\def\bbQ{{\mathbb Q}} 
\def\bbR{{\mathbb R}} 
\def\bbS{{\mathbb S}}
\def\bbT{{\mathbb T}}

% Text colors
\def\tr[#1]{\textcolor{red}{#1}}
\def\tg[#1]{\textcolor{green}{#1}}
\def\tb[#1]{\textcolor{blue}{#1}}
% -----------


% --------- mathcal ------
\def\mcA{{\mathcal A}}
\def\mcC{{\mathcal C}}
\def\mcD{{\mathcal D}}
\def\mcE{{\mathcal E}}
\def\mcF{{\mathcal F}}
\def\mcG{{\mathcal G}}
\def\mcH{{\mathcal H}}
\def\mcI{{\mathcal I}}
\def\mcJ{{\mathcal J}}
\def\mcK{{\mathcal K}}
\def\mcN{{\mathcal N}}
\def\mcO{{\mathcal O}}
\def\mcP{{\mathcal P}}
\def\mcR{{\mathcal R}}
\def\mcS{{\mathcal S}}
\def\mcU{{\mathcal U}}
\def\mcV{{\mathcal V}}
\def\mcX{{\mathcal X}}
\def\mcW{{\mathcal W}}

% math operators
\DeclareMathOperator{\sgn}{sgn}
\DeclareMathOperator{\sym}{sym}
\DeclareMathOperator{\Grad}{Grad}
\DeclareMathOperator*{\assem}{\mathlarger{\mathlarger{\bbA}}}
\DeclareMathOperator*{\argmin}{arg\,min}

%
\def\iden{{\text I}}
\def\dif{{\text{d}}}

\def\bmH{\boldsymbol{\mathcal{H}}}
\def\varep{\varepsilon}
\def\dvar{\dot{\varepsilon}}
\def\grad{\nabla}
\def\grade{\nabla_e}
\def\gradm{\nabla_m}
\newcommand{\bfomega}{\boldsymbol{\omega}}

\def\l{ \left( }
\def\r{ \right) }

% letters
\newcommand{\bfa}{{\bf a}}
\newcommand{\bfA}{{\bf A}}
%
\newcommand{\bfb}{{\bf b}}
\newcommand{\bfB}{{\bf B}}
%
\newcommand{\bfc}{{\bf c}}
\newcommand{\bfC}{{\bf C}}
%
\newcommand{\bfd}{{\bf d}}
\newcommand{\bfD}{{\bf D}}
%
\newcommand{\bfe}{{\bf e}}
\newcommand{\bfE}{{\bf E}}
%
\newcommand{\bff}{{\bf f}}
\newcommand{\bfF}{{\bf F}}
%g
\newcommand{\bfg}{{\bf g}}
\newcommand{\bfG}{{\bf G}}

\newcommand{\bfi}{{\bf i}}
\newcommand{\bfI}{{\bf I}}
%
\newcommand{\bfh}{{\bf h}}
\newcommand{\bfH}{{\bf H}}
%
\newcommand{\bfk}{{\bf k}}
\newcommand{\bfK}{{\bf K}}
%
\newcommand{\bfl}{{\bf l}}
\newcommand{\bfL}{{\bf L}}
% m
\newcommand{\bfm}{{\bf m}}
\newcommand{\bfM}{{\bf M}}
% n
\newcommand{\bfn}{{\bf n}}
\newcommand{\bfN}{{\bf N}}
% p
\newcommand{\bfp}{{\bf p}}
\newcommand{\bfP}{{\bf P}}
% q
\newcommand{\bfq}{{\bf q}}
\newcommand{\bfQ}{{\bf Q}}
% r
\newcommand{\bfr}{{\bf r}}
\newcommand{\bfR}{{\bf R}}
% s
\newcommand{\bfs}{{\bf s}}
\newcommand{\bfS}{{\bf S}}
% t
\newcommand{\bft}{{\bf t}}
\newcommand{\bfT}{{\bf T}}
% u
\newcommand{\bfu}{{\bf u}}
\newcommand{\bfU}{{\bf U}}
% v
\newcommand{\bfv}{{\bf v}}
\newcommand{\bfV}{{\bf V}}
%
\newcommand{\bfx}{{\bf x}}
\newcommand{\bfX}{{\bf X}}

\newcommand{\bfy}{{\bf y}}
\newcommand{\bfY}{{\bf Y}}

\newcommand{\bfw}{{\bf w}}
\newcommand{\bfW}{{\bf W}}

\newcommand{\bfz}{{\bf z}}
\newcommand{\bfZ}{{\bf Z}}

% simbolo pormil
\def\permil{\textperthousand}


\newcommand{\bfvarep}{\boldsymbol{\varepsilon}}
\newcommand{\bfsig}{\boldsymbol{\sigma}}
\newcommand{\bfmu}{\boldsymbol{\mu}}
\newcommand{\bftau}{\boldsymbol{\tau}}
\newcommand{\bfphi}{\boldsymbol{\phi}}
\newcommand{\bfPhi}{\boldsymbol{\Phi}}
\newcommand{\bfLam}{\boldsymbol{\Lambda}}
\newcommand{\bfnu}{\boldsymbol{\nu}}

\newcommand{\bsone}{\boldsymbol{1}}
\newcommand{\bszer}{\boldsymbol{0}}

\def\R{{\mathbb R}}
\def\x{{\mathbf x}}
\def\U{{\textbf{U}}}
\def\mF{{\mathcal F}}
\def\P{\mathcal{P}}
\def\mcL{\mathcal{L}}

\DeclareMathOperator{\tra}{\text{Tr}}


\renewcommand{\leq}{\leqslant}
\renewcommand{\geq}{\geqslant}

\newcommand{\tq}{\;\;|\;\;}

\DeclareMathOperator{\vol}{vol}
\DeclareMathOperator{\inv}{inv}
\DeclareMathOperator{\area}{\acute{a}rea}
\DeclareMathOperator{\sig}{sig}
\DeclareMathOperator{\diam}{diam}

\newcommand{\abs}[1]{\lvert#1\rvert}
\newcommand{\norm}[1]{\lVert#1\rVert}
\newcommand{\Abs}[1]{\left\lvert#1\right\rvert}
\newcommand{\Norm}[1]{\left\lVert#1\right\rVert}


% --------------------------------------------
% Numeracion de secciones (hasta subsection)
\setcounter{secnumdepth}{2}

\addto\captionsspanish{%
	\renewcommand\contentsname{Lista de contenidos}%
}


\usepackage{url}
\urldef\urlPrimeraEdicion\url{https://www.colibri.udelar.edu.uy/jspui/bitstream/20.500.12008/22106/1/Bazzano_P%c3%a9rezZerpa_Introducci%c3%b3n_al_An%c3%a1lisis_No_Lineal_de_Estructuras_2017.pdf}
